% ============================================================================
% Резюме / Abstract
% ============================================================================
\section*{Резюме}

Настоящата дипломна работа представя проектирането, реализацията и експерименталната оценка на Coroute --- високопроизводителна C++ библиотека за изграждане на уеб приложения, базирана на съвременните възможности на стандарта C++20. Основната цел на изследването е да се демонстрира, че комбинацията от stackless корутини, операционно-специфични асинхронни I/O механизми и алгоритмично оптимизирано маршрутизиране може да постигне производителност, съизмерима с водещите индустриални решения, при значително опростен програмен модел.

Архитектурата на Coroute се основава на няколко ключови иновации. Първо, изпълнителният модел използва типа \texttt{Task<T>}, който реализира lazy корутини с поддръжка на detached изпълнение и кооперативно прекратяване. Този модел елиминира традиционните недостатъци на callback-базираното асинхронно програмиране, като запазва ефективността на неблокиращия I/O. Второ, мрежовият слой абстрахира платформено-специфичните механизми --- IOCP за Windows и io\_uring за Linux --- зад унифициран интерфейс, позволявайки zero-copy операции и batching на системни извиквания. Трето, маршрутизирането използва DFA-базиран алгоритъм, публикуван в „Matching Text from Start to Finish Against Multiple Regular Expressions" \cite{stankov2024regex}, който постига O(N) времева сложност спрямо дължината на URL-а, независимо от броя на регистрираните маршрути.

Библиотеката поддържа множество протоколи върху един порт чрез ALPN негоциация: HTTP/1.1 с keep-alive, HTTP/2 с HPACK компресия и stream мултиплексиране, и WebSocket за двупосочна комуникация в реално време. Типово-безопасното извличане на параметри използва C++20 concepts за compile-time валидация, елиминирайки цял клас runtime грешки.

Експерименталната оценка, проведена на система с AMD Ryzen 5 3600 (6 ядра, 12 нишки) и до 1024 конкурентни клиента, демонстрира throughput от 1,378,578 заявки в секунда за прости HTTP отговори, с медианна латентност от 212 микросекунди и 0\% error rate. Сравнителният анализ с Crow, Boost.Beast и Drogon показва, че Coroute постига сравнима или по-добра производителност при значително по-прост API.

Научният принос на работата включва: интеграция на DFA-базирано маршрутизиране в пълноценен уеб библиотека; корутинен изпълнителен модел, специално проектиран за мрежови приложения; и демонстрация на практическата приложимост на подхода чрез production-scale примерно приложение.

\vspace{1cm}

\textbf{Ключови думи:} C++20, корутини, уеб сървър, HTTP/2, WebSocket, IOCP, io\_uring, DFA маршрутизиране, типова безопасност, асинхронен I/O

\vspace{2cm}

\section*{Abstract}

This thesis presents the design, implementation, and experimental evaluation of Coroute --- a high-performance C++ library for building web applications based on modern C++20 features. The primary objective is to demonstrate that the combination of stackless coroutines, OS-native asynchronous I/O mechanisms, and algorithmically optimized routing can achieve performance comparable to leading industrial solutions while significantly simplifying the programming model.

The architecture of Coroute is founded on several key innovations. The execution model employs the \texttt{Task<T>} type, implementing lazy coroutines with support for detached execution and cooperative cancellation. This model eliminates the traditional drawbacks of callback-based asynchronous programming while preserving the efficiency of non-blocking I/O. The network layer abstracts platform-specific mechanisms --- IOCP for Windows and io\_uring for Linux --- behind a unified interface, enabling zero-copy operations and syscall batching. Routing utilizes a DFA-based algorithm, published in ``Matching Text from Start to Finish Against Multiple Regular Expressions'' \cite{stankov2024regex}, achieving O(N) time complexity relative to URL length, independent of the number of registered routes.

The library supports multiple protocols on a single port via ALPN negotiation: HTTP/1.1 with keep-alive, HTTP/2 with HPACK compression and stream multiplexing, and WebSocket for bidirectional real-time communication. Type-safe parameter extraction leverages C++20 concepts for compile-time validation, eliminating an entire class of runtime errors.

Experimental evaluation on a system with AMD Ryzen 5 3600 (6 cores, 12 threads) and up to 1024 concurrent clients demonstrates throughput of 1,378,578 requests per second for simple HTTP responses, with median latency of 212 microseconds and 0\% error rate. Comparative analysis with Crow, Boost.Beast, and Drogon shows that Coroute achieves comparable or superior performance with a significantly simpler API.

The scientific contributions include: integration of DFA-based routing into a full-featured web framework; a coroutine execution model specifically designed for network applications; and demonstration of practical applicability through a production-scale example application.

\vspace{1cm}

\textbf{Keywords:} C++20, coroutines, web server, HTTP/2, WebSocket, IOCP, io\_uring, DFA routing, type safety, asynchronous I/O
