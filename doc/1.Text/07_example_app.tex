% ============================================================================
% Глава 7: Примерно приложение
% ============================================================================
\section{Примерно приложение}

Теоретичните концепции и архитектурните решения, представени в предходните глави, придобиват практическа стойност едва когато бъдат приложени в реалистичен контекст. Настоящата глава представя Task Dashboard --- пълноценна система за управление на задачи в реално време, която служи като референтна имплементация за изграждане на production-ready уеб приложения с Coroute.

Примерното приложение е проектирано с две основни цели. Първо, да демонстрира интеграцията на всички ключови компоненти на Coroute в едно цялостно решение: REST API, WebSocket комуникация, автентикация, middleware верига и server-side rendering. Второ, да илюстрира best practices и архитектурни patterns, които могат да се прилагат при изграждане на реални уеб приложения.

\subsection{Описание на приложението}

Task Dashboard е уеб приложение за колаборативно управление на задачи в екипна среда. Системата позволява на потребителите да създават, редактират и изтриват задачи, като промените се отразяват в реално време при всички свързани клиенти чрез WebSocket. Този use case е избран, защото изисква комбинация от традиционни HTTP операции (CRUD) и real-time комуникация, което е типично за съвременните уеб приложения.

Приложението демонстрира следните функционалности на Coroute: REST API с пълен набор от CRUD операции за задачи и потребители, WebSocket за broadcast на актуализации в реално време, server-side rendering с HTML шаблони чрез inja библиотеката, session-based автентикация с cookie management, middleware верига за логване на заявки и защита на маршрути, и обслужване на статични файлове (CSS, JavaScript) с zero-copy I/O.

\subsection{Структура на проекта}

Проектът следва стандартна структура за C++ уеб приложения, с ясно разделение между различните слоеве на приложението. Source файловете са организирани по функционалност, а не по тип, което улеснява навигацията и поддръжката.

\begin{lstlisting}[basicstyle=\small\ttfamily, frame=none, numbers=none]
Project/
+-- src/
|   +-- main.cpp              # Entry point
|   +-- app/                  # Server configuration
|   |   +-- config.hpp        # Configuration
|   |   +-- server.hpp        # Server setup
|   +-- middleware/           # Middleware components
|   |   +-- auth.cpp          # Authentication
|   |   +-- logging.cpp       # Request logging
|   +-- handlers/             # Route handlers
|   |   +-- pages.cpp         # HTML pages
|   |   +-- api/              # REST endpoints
|   |   |   +-- tasks.cpp     # Task CRUD
|   |   |   +-- users.cpp     # User management
|   |   +-- websocket/        # Real-time hub
|   |       +-- task_hub.cpp  # WebSocket handler
|   +-- models/               # Data structures
|   |   +-- task.hpp          # Task model
|   |   +-- user.hpp          # User model
|   +-- services/             # Business logic
|       +-- task_service.hpp  # Task operations
|       +-- user_service.hpp  # User operations
+-- templates/                # HTML templates
|   +-- layout.html           # Base layout
|   +-- pages/                # Page templates
|       +-- index.html        # Dashboard
|       +-- login.html        # Login page
+-- static/                   # Static assets
|   +-- css/                  # Stylesheets
|   +-- js/                   # JavaScript
+-- tests/                    # Unit tests
\end{lstlisting}

\subsection{Конфигурация}

\subsubsection{Config клас}

\begin{lstlisting}[language=C++, basicstyle=\small\ttfamily, caption={Configuration class}]
struct Config {
    uint16_t port = 8080;
    size_t thread_count = 4;
    std::string static_path = "static";
    std::string template_path = "templates";
    
    // TLS configuration (optional)
    std::optional<TlsConfig> tls;
    
    static Config from_env() {
        Config config;
        
        if (auto port = std::getenv("PORT")) {
            config.port = static_cast<uint16_t>(std::stoi(port));
        }
        
        if (auto threads = std::getenv("THREADS")) {
            config.thread_count = std::stoul(threads);
        }
        
        return config;
    }
};
\end{lstlisting}

\subsubsection{Server setup}

\begin{lstlisting}[language=C++, basicstyle=\small\ttfamily, caption={Server initialization}]
class Server {
    coroute::App app_;
    Config config_;
    
public:
    explicit Server(const Config& config) : config_(config) {
        app_.threads(config.thread_count);
        
        // Setup middleware
        setup_middleware();
        
        // Setup routes
        setup_routes();
        
        // Setup WebSocket
        setup_websocket();
        
        // Setup static files
        app_.static_files("/static", config.static_path);
    }
    
    void run() {
        app_.run(config_.port);
    }
    
    void stop() {
        app_.shutdown({
            .drain_timeout = std::chrono::seconds(30),
            .force_close_after_timeout = true
        });
    }
};
\end{lstlisting}

\subsection{Middleware}

\subsubsection{Logging middleware}

\begin{lstlisting}[language=C++, basicstyle=\small\ttfamily, caption={Request logging middleware}]
Middleware logging_middleware() {
    return [](Request& req, Next next) -> Task<Response> {
        auto start = std::chrono::steady_clock::now();
        
        // Log request
        std::cout << req.method_string() << " " << req.path() << std::endl;
        
        // Call next middleware/handler
        auto resp = co_await next(req);
        
        // Log response time
        auto elapsed = std::chrono::steady_clock::now() - start;
        auto ms = std::chrono::duration_cast<
            std::chrono::milliseconds>(elapsed).count();
        
        std::cout << "  -> " << resp.status_code() 
                  << " (" << ms << "ms)" << std::endl;
        
        co_return resp;
    };
}
\end{lstlisting}

\subsubsection{Authentication middleware}

\begin{lstlisting}[language=C++, basicstyle=\small\ttfamily, caption={Authentication middleware}]
Middleware auth_middleware(UserService& user_service) {
    return [&user_service](Request& req, Next next) -> Task<Response> {
        // Check for session cookie
        auto session_id = req.cookie("session_id");
        if (!session_id) {
            co_return Response::unauthorized("Not authenticated");
        }
        
        // Validate session
        auto user = user_service.get_by_session(*session_id);
        if (!user) {
            co_return Response::unauthorized("Invalid session");
        }
        
        // Store user in request context
        req.set_context("user", *user);
        
        // Continue to handler
        co_return co_await next(req);
    };
}
\end{lstlisting}

\subsection{REST API handlers}

\subsubsection{Task CRUD}

\begin{lstlisting}[language=C++, basicstyle=\small\ttfamily, caption={Task API handlers}]
void setup_task_routes(App& app, TaskService& task_service, 
                       TaskHub& task_hub) {
    // List all tasks
    app.get("/api/tasks", [&](Request& req) -> Task<Response> {
        auto tasks = task_service.get_all();
        co_return Response::json(tasks);
    });
    
    // Get task by ID
    app.get<int>("/api/tasks/{id}", 
        [&](int id, Request& req) -> Task<Response> {
        auto task = task_service.get_by_id(id);
        if (!task) {
            co_return Response::not_found("Task not found");
        }
        co_return Response::json(*task);
    });
    
    // Create task
    app.post("/api/tasks", [&](Request& req) -> Task<Response> {
        auto body = req.json<TaskCreateRequest>();
        if (!body) {
            co_return Response::bad_request("Invalid JSON");
        }
        
        auto task = task_service.create(*body);
        
        // Broadcast to WebSocket clients
        task_hub.broadcast({
            .type = "task_created",
            .payload = task
        });
        
        co_return Response::created(task);
    });
    
    // Update task
    app.put<int>("/api/tasks/{id}", 
        [&](int id, Request& req) -> Task<Response> {
        auto body = req.json<TaskUpdateRequest>();
        if (!body) {
            co_return Response::bad_request("Invalid JSON");
        }
        
        auto task = task_service.update(id, *body);
        if (!task) {
            co_return Response::not_found("Task not found");
        }
        
        // Broadcast update
        task_hub.broadcast({
            .type = "task_updated",
            .payload = *task
        });
        
        co_return Response::json(*task);
    });
    
    // Delete task
    app.del<int>("/api/tasks/{id}", 
        [&](int id, Request& req) -> Task<Response> {
        if (!task_service.remove(id)) {
            co_return Response::not_found("Task not found");
        }
        
        // Broadcast deletion
        task_hub.broadcast({
            .type = "task_deleted",
            .payload = {{"id", id}}
        });
        
        co_return Response::no_content();
    });
}
\end{lstlisting}

\subsection{WebSocket handler}

\subsubsection{TaskHub клас}

\begin{lstlisting}[language=C++, basicstyle=\small\ttfamily, caption={WebSocket TaskHub}]
class TaskHub {
    std::mutex mutex_;
    std::set<WebSocketConnection*> clients_;
    
public:
    void add_client(WebSocketConnection* client) {
        std::lock_guard lock(mutex_);
        clients_.insert(client);
    }
    
    void remove_client(WebSocketConnection* client) {
        std::lock_guard lock(mutex_);
        clients_.erase(client);
    }
    
    void broadcast(const json& message) {
        std::lock_guard lock(mutex_);
        auto data = message.dump();
        
        for (auto* client : clients_) {
            // Fire-and-forget send
            client->send(data);
        }
    }
};
\end{lstlisting}

\subsubsection{WebSocket route}

\begin{lstlisting}[language=C++, basicstyle=\small\ttfamily, caption={WebSocket route setup}]
void setup_websocket(App& app, TaskHub& task_hub) {
    app.websocket("/ws", [&task_hub](
        std::unique_ptr<WebSocketConnection> ws) -> Task<void> {
        
        auto* ws_ptr = ws.get();
        task_hub.add_client(ws_ptr);
        
        // RAII cleanup
        struct Cleanup {
            TaskHub& hub;
            WebSocketConnection* client;
            ~Cleanup() { hub.remove_client(client); }
        } cleanup{task_hub, ws_ptr};
        
        // Message loop
        while (true) {
            auto msg = co_await ws->receive();
            if (!msg) break;  // Connection closed
            
            if (msg->type == WebSocketMessage::Text) {
                // Handle client messages (e.g., subscriptions)
                auto data = json::parse(msg->data);
                // Process message...
            }
        }
    });
}
\end{lstlisting}

\subsection{HTML шаблони}

\subsubsection{Template rendering}

\begin{lstlisting}[language=C++, basicstyle=\small\ttfamily, caption={Template rendering}]
app.get("/", [&](Request& req) -> Task<Response> {
    auto tasks = task_service.get_all();
    auto stats = task_service.get_statistics();
    
    auto html = render_template("pages/index.html", {
        {"tasks", tasks},
        {"stats", stats},
        {"user", req.context<User>("user")}
    });
    
    co_return Response::html(html);
});
\end{lstlisting}

\subsubsection{Примерен шаблон}

\begin{lstlisting}[language=HTML, basicstyle=\small\ttfamily, caption={Dashboard template}]



<div class="dashboard">
    <h1>Task Dashboard</h1>
    
    <div class="stats">
        <div class="stat">
            <span class="value">{{ stats.total }}</span>
            <span class="label">Total Tasks</span>
        </div>
        <div class="stat">
            <span class="value">{{ stats.completed }}</span>
            <span class="label">Completed</span>
        </div>
    </div>
    
    <ul class="task-list" id="tasks">
        
        <li data-id="{{ task.id }}">
            <span class="title">{{ task.title }}</span>
            <span class="status">{{ task.status }}</span>
        </li>
        
    </ul>
</div>

\end{lstlisting}

\subsection{JavaScript клиент}

\begin{lstlisting}[basicstyle=\small\ttfamily, caption={WebSocket client (JavaScript)}]
const ws = new WebSocket(`ws://${location.host}/ws`);

ws.onmessage = (event) => {
    const { type, payload } = JSON.parse(event.data);
    
    switch (type) {
        case 'task_created':
            addTaskToList(payload);
            break;
        case 'task_updated':
            updateTaskInList(payload);
            break;
        case 'task_deleted':
            removeTaskFromList(payload.id);
            break;
    }
};

ws.onclose = () => {
    // Reconnect after delay
    setTimeout(() => location.reload(), 3000);
};
\end{lstlisting}

\subsection{Entry point}

\begin{lstlisting}[language=C++, basicstyle=\small\ttfamily, caption={main.cpp}]
#include "app/config.hpp"
#include "app/server.hpp"
#include <csignal>

namespace {
    project::Server* g_server = nullptr;
    
    void signal_handler(int signal) {
        if (signal == SIGINT || signal == SIGTERM) {
            std::cout << "\nShutting down...\n";
            if (g_server) {
                g_server->stop();
            }
        }
    }
}

int main() {
    try {
        auto config = project::Config::from_env();
        
        project::Server server(config);
        g_server = &server;
        
        std::signal(SIGINT, signal_handler);
        std::signal(SIGTERM, signal_handler);
        
        server.run();
        
        return 0;
    } catch (const std::exception& e) {
        std::cerr << "Fatal error: " << e.what() << "\n";
        return 1;
    }
}
\end{lstlisting}

\subsection{Анализ на архитектурните решения}

Примерното приложение илюстрира няколко ключови архитектурни решения, които са характерни за Coroute и го отличават от други библиотеки.

\subsubsection{Корутинен модел за WebSocket}

WebSocket handler-ът демонстрира елегантността на корутинния модел. Вместо callback-базиран подход с регистрация на event handlers, кодът е структуриран като последователен цикъл с \texttt{co\_await}. Това значително опростява управлението на състоянието и обработката на грешки. RAII pattern-ът за cleanup гарантира, че клиентът се премахва от hub-а дори при неочаквано прекъсване на връзката.

\subsubsection{Типово-безопасни route параметри}

Маршрутите като \texttt{app.get<int>("/api/tasks/\{id\}", ...)} демонстрират типово-безопасното извличане на параметри. Параметърът \texttt{id} се конвертира автоматично към \texttt{int} и се подава на handler-а като типизиран аргумент. Грешки при конвертиране се обработват автоматично от библиотеката, връщайки 400 Bad Request.

\subsubsection{Middleware композиция}

Middleware веригата позволява модулно добавяне на cross-cutting concerns. Logging middleware-ът измерва времето за обработка на всяка заявка, а authentication middleware-ът защитава определени маршрути. Middleware компонентите са независими и могат да се комбинират в произволен ред.

\subsection{Сравнение с други библиотеки}

За да се оцени практическата стойност на Coroute, е полезно да се сравни сложността на имплементация на Task Dashboard с други библиотеки.

При Express.js (Node.js), подобно приложение би изисквало приблизително същия обем код, но с callback-базиран или Promise-базиран асинхронен модел. JavaScript не предоставя compile-time type checking, което означава, че грешки в типовете на параметрите се откриват едва при runtime.

При Drogon (C++), кодът би бил подобен по структура, но асинхронните операции биха използвали callbacks или частична coroutine поддръжка. Drogon предоставя ORM интеграция, която липсва в Coroute, но това е съзнателно архитектурно решение за запазване на модулност.

При Flask (Python), приложението би било значително по-кратко поради динамичната природа на езика, но производителността би била с порядъци по-ниска. WebSocket поддръжката изисква допълнителни библиотеки като Flask-SocketIO.

\subsection{Заключение}

Task Dashboard демонстрира, че Coroute е подходящ за изграждане на пълноценни production-ready уеб приложения. Корутинният модел опростява асинхронния код, типово-безопасните параметри подобряват надеждността, а middleware системата позволява модулна архитектура. Примерното приложение може да служи като отправна точка за разработчици, които искат да използват Coroute в реални проекти.
